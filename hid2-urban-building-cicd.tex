\documentclass[runningheads]{llncs}
\usepackage{fullpage}
\usepackage{graphicx} % for including images

\begin{document}
\title{Enhancing Urban Building Simulations in HiDALGO2 through CI/CD Innovations}
\author{Vincent Chabannes \inst{1}\orcidID{0000-1111-2222-3333} and Christophe Prud'homme\inst{1}\orcidID{0000-0003-2287-2961}}
\institute{Cemosis, IRMA UMR 7501, University of Strasbourg, CNRS\\ 
\email{\{vincent.chabannes,christophe.prudhomme\}@cemosis.fr}}
\authorrunning{V. Chabannes and C. Prud'homme}


\maketitle

\begin{abstract}
The building sector in the European Union significantly impacts energy consumption and greenhouse gas emissions. The EU's Horizon 2050 initiative sets ambitious goals to reduce these impacts through enhanced building renovation rates. The HiDALGO2 project supports this initiative by developing high-performance computing solutions, specifically through the Urban Building pilot application which utilizes advanced CI/CD methodologies to streamline simulation and deployment across various computational platforms.
\end{abstract}

\section{Introduction}
The building sector accounts for approximately 40\% of final energy consumption and 36\% of greenhouse gas emissions within the European Union. In response, the EU has established ambitious targets under the Horizon 2050 framework to double energy renovation rates over the next decade, highlighting the need for innovative solutions to drive these initiatives forward. The HiDALGO2 project, with its focus on high-performance computing and advanced simulations, is at the forefront of tackling this challenge, particularly through its Urban Building pilot application.

\section{Objectives of the HiDALGO2 Urban Building Pilot}
The Urban Building pilot in HiDALGO2 aims to leverage high-performance computing to enhance urban simulations for better energy management and air quality assessment. This section outlines the specific objectives and expected impacts of the project.

\subsection{Predictive Simulations}
Advanced simulation tools are employed to predict energy consumption, thermal comfort, and indoor air quality across both the building and urban scales. These simulations support detailed analysis at the individual building level and extend to broader urban environments, influencing urban planning and policy-making.

\subsection{Integration with Environmental Models}
Integrating building simulations with urban air pollution models enables the project to assess the environmental impact of building stocks comprehensively. This integration improves the predictive accuracy of the simulations by incorporating real-time data such as wind speed and solar radiation, enhancing the models' responsiveness to environmental conditions.

\subsection{Impact and Implementation}
Implementing these objectives will facilitate more informed urban planning decisions, support policy development for energy efficiency, and contribute to reducing urban greenhouse gas emissions. The project also focuses on enhancing the interaction between different environmental models to provide a holistic view of urban ecosystems.

\section{The Role of CI/CD in Urban Building Simulations}
Continuous Integration (CI) and Continuous Deployment (CD) are pivotal in the development and operation of the Urban Building pilot. These practices enable efficient management of complex simulation software, ensuring seamless integration and deployment across various computational infrastructures.

\subsection{Continuous Integration: Ensuring Code Quality and Reliability}
CI practices in HiDALGO2 handle complex dependencies and computational requirements specific to urban building simulations. Key features include automated testing, code quality checks, and automated builds, ensuring that the software remains robust and maintainable.

\subsection{Continuous Deployment: Streamlining Software Delivery}
CD practices are tailored to manage deployment complexities across different HPC environments using containerization through Apptainer. This approach facilitates consistent, reproducible deployments, optimizing performance and resource utilization.

\section{Conclusion}
CI/CD practices in HiDALGO2 are vital for advancing urban building simulations, supporting the EU's energy efficiency goals, and contributing to the reduction of greenhouse gas emissions in the urban sector.

\bibliographystyle{splncs04}
\bibliography{mybib}

\end{document}